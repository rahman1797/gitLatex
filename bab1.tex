%!TEX root = ./template-skripsi.tex
%-------------------------------------------------------------------------------
% 								BAB I
% 							LATAR BELAKANG
%-------------------------------------------------------------------------------

\chapter{LATAR BELAKANG}

\section{Latar Belakang}
Manajemen merupakan seluruh upaya dalam mendayagunakan sumber daya untuk mencapai tujuan secara efektif dan efisien \cite{rifai}. Manajemen juga dapat dikatakan sebagai seni atau ilmu dalam mengatur yang memanfaatkan kemampuan orang lain untuk menyelesaikan suatu pekerjaan dengan fungsi-fungsi dari manajemen terhadap usaha yang dilakukan oleh orang lain. Fungsi-fungsi tersebut secara umum dapat berupa tindakan perencanaan, pengorganisasian, Pengarahan dan Pengawasan \cite{rachman}.

Organisasi merupakan sistem yang terpadu yang didalamnya terdapat subsistem dan komponen-komponen yang saling berhubungan. Setiap hubungan yang terjadi merupakan kerjasama diantara subsistem yang ada, sehingga ada saling ketergantungan yang kuat secara internal dan hubungan yang terpadu secara eksternal. Hubungan eksternal itu merupakan bagian dari kenyataan organisasi yang berkaitan dengan lingkungan masyarakat dan elemen lainnya yang mendukung tercapainya tujuan organisasi \cite{priansa}. Definisi tersebut menekankan bahwa organisasi dibuat untuk mewujudkan kebutuhan dan kepentingan masyarakat.

Didalam sebuah organisasi pada umumnya terdiri dari berbagai struktur kepengurusan yang memiliki fungsi dan peranan masing-masing. Seluruh anggota organisasi memerlukan kemampuan dalam memanajemen perannya masing-masing. Fungsi pengorganisasian dalam manajemen tersebut harus memperhatikan kemampuan dari setiap anggota untuk membagi peran dan tugas agar tujuan dari sebuah organisasi dapat tercapai secara efektif. Tanpa adanya manajemen yang baik, pengelolaan dan pengorganisasian tidak akan berjalan secara optimal, sehingga manajemen sangat dibutuhkan di dalam organisasi agar segala tujuan dapat tercapai dengan baik.

Universitas Negeri Jakarta memiliki dua macam organisasi, yakni bersifat organisasi pemerintahan atau yang biasa dikenal dengan sebutan Organisasi Pemerintahan Mahasiswa (opmawa) dan organisasi publik yang dikenal dengan sebutan Unit Kegiatan Mahasiswa (UKM). Opmawa sebagai organisasi pemerintahan merupakan organisasi yang berfokus pada kepentingan birokrasi dan mahasiswa. Pengurus organisasi memiliki kewajiban untuk menghayati keadaan ini sebagai sebuah obligasi sehingga responsif terhadap kepentingan umum.

Opmawa terbagi atas tiga tingkatan, yaitu tingkat Universitas, Fakultas, dan Program Studi. Masing-masing tingkatan tersebut terdapat dua Lembaga, yakni Lembaga eksekutif (BEM) dan Lembaga legislatif. Lembaga eksekutif memiliki tugas utama sebagai Lembaga yang melaksanakan pelayanan secara langsung terhadap mahasisawa dan Lembaga legislatif memiliki tugas utama sebagai Lembaga yang mengawasi dan menampung aspirasi mahasiswa. 

Opmawa sebagai organisasi memiliki program kerja yang meliputi berbagai rencana kegiatan yang akan dilaksanakan. Kegiatan yang dilaksanakan dapat berupa masa pengenalan akademik, pelatihan kepemimpinan mahasiswa, malam keakraban, seminar, \emph{workshop}, festival, dan lain-lain. Banyaknya kegiatan tersebut dibutuhkan kemampuan manajerial yang baik seperti manajemen keuangan, sumber daya manusia, informasi, strategi, dan juga teknis operasi yang dibutuhkan di lapangan. Pentingnya manajemen terhadap kegiatan yang akan dilaksanakan dapat membantu terlaksananya pelayanan yang maksimal dan mencapai tujuan yang diinginkan.

Faktanya didalam sebuah organisasi, khususnya ketika mengadakan suatu kegiatan, sering terjadi \emph{miss communication} antar pengurus. Permasalahan yang sering terjadi dikarenakan sistem manajemen peran yang kurang baik, seperti masalah komunikasi antar anggota, kedisiplinan, pengelolaan keuangan, logistik, data kesekretariatan, dan permasalahan-permasalahan umum lainnya. Maka dibutuhkan sebuah sistem yang dapat meminimalisir terjadinya permasalahan tersebut.

Untuk meningkatkan kualitas atau kinerja opmawa Fakultas Matematika dan Ilmu Pengetahuan Alam (FMIPA) serta untuk meminimalisir terjadinya permasalahan yang terjadi, diperlukan sebuah sistem informasi manajemen program kerja yang dapat membantu dalam pelaksanaan perencanaan dan pengorganisasian terhadap kegiatan-kegiatan yang akan dilaksanakan. Sistem informasi tesebut diharapkan dapat membantu dalam pengelolaan data keuangan, evaluasi kegiatan, memonitor kinerja dari setiap anggota atau peran, serta menjadi sebuah sistem yang dapat menyatukan komunikasi antar anggota. 

Berdasarkan permasalahan tersebut penulis memberikan alternatif dengan merancang sebuah sistem informasi manajemen berbasis \emph{website} yang mampu membantu kebutuhan pengelolaan manajemen program kerja opmawa di Fakultas Matematika dan Ilmu Pengetahuan Alam. Pada sistem informasi manajemen program kerja opmawa memfokuskan sistem dalam pengolahan data terkait program kerja dan mengolahnya menjadi informasi yang dapat membantu dalam pelaksanaan manajerial organisasi. Fitur-fitur yang memungkinkan untuk diberikan antara lain seperti pengolahan data keuangan, evaluasi program kerja beserta penilaian yang diberikan oleh Lembaga legislatif, keaktifan anggota organisasi, \emph{to do list}, \emph{activity reminder}, dan lain-lain.

\section{Perumusan Masalah}
Adapun rumusan masalah berdasarkan pada latar belakang yang telah dipaparkan adalah sebagai berikut :
\begin{enumerate}
	\item Bagaimana alur pengembangan \emph{website} yang sesuai dengan \textit{System Development Life Cycle} (SDLC) pada aplikasi sistem informasi manajemen program kerja opmawa ?
	
	\item Bagaimana cara mengimplementasikan aplikasi sistem informasi manajemen program kerja opmawa berbasis \emph{website} dengan menggunakan \emph{framework} Codeigniter ?
\end{enumerate}

\section{Pembatasan Masalah}
Adapun batasan-batasan masalah pada penelitian ini agar lebih terarah dan sesuai dengan yang diharapkan adalah sebagai berikut :
\begin{enumerate}
	\item Sistem informasi yang dirancang berkaitan dengan kebutuhan informasi program kerja opmawa pada tingkat Fakultas.
	\item Model pengembangan yang digunakan untuk mengembangkan sistem menggunakan model Spiral.
	\item \textit{Framework} yang digunakan untuk keperluan \textit{back-end} menggunakan framework Codeigniter dan untuk keperluan \textit{front-end} menggunakan framework Bootstrap.
\end{enumerate}

\section{Tujuan Penelitian}
Adapun tujuan penelitian aplikasi sistem informasi manajemen program kerja opmawa berbasis \emph{website} ini adalah sebagai berikut : 
\begin{enumerate}
	\item Untuk mengetahui bagaimana rancangan desain aplikasi sistem informasi manajemen program kerja opmawa berbasis \emph{website}.
	\item Untuk mengetahui cara mengembangkan aplikasi sistem informasi manajemen program kerja opmawa berbasis \emph{website} dengan menggunakan \emph{framework} Codeigniter
\end{enumerate}

\section{Manfaat Penelitian}
Adapun manfaat yang diharapkan dari penelitian ini adalah sebagai berikut :
	\begin{enumerate}
		\item Menjadi alternatif dalam menghadapi permasalahan manajemen yang terdapat pada organisasi pemerintahan mahasiswa Fakultas Matematika dan Ilmu Pengetahuan Alam.
		\item Sebagai jembatan komunikasi antar lembaga organisasi dan periode kepengurusan dalam merencanakan, menjalankan, dan mengevaluasi program kerja.
		\item Sebagai gambaran dan bahan referensi untuk penelitian dalam membangun sistem informasi manajemen program kerja.   	
	\end{enumerate}
		
% Baris ini digunakan untuk membantu dalam melakukan sitasi
% Karena diapit dengan comment, maka baris ini akan diabaikan
% oleh compiler LaTeX.
\begin{comment}
\bibliography{daftar-pustaka}
\end{comment}
